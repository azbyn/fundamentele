% Created 2019-03-15 Fri 00:00
% Intended LaTeX compiler: pdflatex
\documentclass[11pt]{article}
\usepackage[utf8]{inputenc}
\usepackage[T1]{fontenc}
\usepackage{graphicx}
\usepackage{grffile}
\usepackage{longtable}
\usepackage{wrapfig}
\usepackage{rotating}
\usepackage[normalem]{ulem}
\usepackage{amsmath}
\usepackage{textcomp}
\usepackage{amssymb}
\usepackage{capt-of}
\usepackage{hyperref}
\usepackage{minted}
\usepackage{geometry}\geometry{a4paper,left=20mm,right=20mm,top=20mm,bottom=30mm}
\usepackage{mathtools,accents,graphicx,tipa}
\date{}
\title{}
\hypersetup{
 pdfauthor={},
 pdftitle={},
 pdfkeywords={},
 pdfsubject={},
 pdfcreator={Emacs 26.1 (Org mode 9.2)}, 
 pdflang={English}}
\begin{document}


\makeatletter
\DeclareFontFamily{U}{tipa}{}
\DeclareFontShape{U}{tipa}{m}{n}{<->tipa10}{}
\newcommand{\ark@char}{{\usefont{U}{tipa}{m}{n}\symbol{62}}}%

\newcommand{\ark}[1]{\mathpalette\ark@ark{#1}}

\newcommand{\ark@ark}[2]{%
  \sbox0{$\m@th#1#2$}%
  \vbox{
    \hbox{\resizebox{\wd0}{\height}{\ark@char}}
    \nointerlineskip
    \box0
  }%
}
\makeatother

Demonstrați că $\overline{A \cup B} = \overline{A} \cup \overline{B}.$

\vspace{7pt}

Știm: $\forall A, B \in X$ \\
(1) $\overline{cA} = c\mathring{A}$ \\
(2) $c(A \cap B) = cA \cup cB$\\
(3) $\mathring{\ark{A \cap B}} = \mathring{A} \cap \mathring{B}$\\

Fie $Y, Z \subseteq X$.

Fie $A = cY, Z = cB$.

Demonstrăm că  $\overline{A \cup B} = \overline{A} \cup \overline{B}$.
\vspace{7pt}

$A \cup B = cY \cup cZ  \stackrel{(2)}{=} c(Y \cap Z) \implies\ \overline{ A \cup B } = \overline{ c(Y \cap Z) }$
$ \stackrel{(1)}{=} c(\mathring{\ark{ Y \cap Z }}) \stackrel{(3)}{=} c(\mathring{Y} \cap  \mathring{Z})$
$ \stackrel{(2)}{=} c\mathring{Y} \cup c\mathring{Z} = \\ = \overline{cY} \cup \overline{cZ} = \overline{A} \cup \overline{B}.$


q.e.d.
\end{document}
